\problemname{One Last Stand!}
\illustration{.5}{hydra}{\href{https://www.goodfreephotos.com/albums/vector-images/hydra-silhouette-vector-clipart.png}{GoodFreePhotos.com}, public domain}

\noindent You and your convoy thankfully managed to get through the the Mist Dragon's --- Niebla --- domain unharmed. All thanks to your
talented tactician. The mist seems to have settled and now doesn't bother you as much. \\

\noindent But, as you continue walking, you notice that the mist gets thicker. You look back and can barely make out the silhoette,
of the members of your convoy, when suddenly you feel your feet sink into some water. You immediately face forward and notice
a grand lake before you. \\

\noindent As your eyes adjust to the thickness of the mist, you see a huge silhouette approaching you. As it approaches you hear
it sound become ever more vicous. In the blink of an eye, standing there before you is the Mist Dragon Niebla, the infamous hydra! \\

\noindent The mist dragon immediately attacks you, luckily you manage to block just in time. Suddenly Niebla loses a
few heads. But that can't be right, old stories and folklore have taught you that hydras only grow heads when they are
attacked. Before you can finish this thought Niebla attacks again as you manage to defend just in the nick of time .
This time Niebla has grew some heads. The madness, this beast does not follow conventional logic! You decide to take the
 brunt of the Niebla's next attck so you can study this pattern. Niebla attacks and not too soon after you manage to
 cut off a head, this time Niebla seems to have lost some more in addition to the one you just cut off. A
thought crosses your head,  "Maybe Niebla is too unstable to be able to properly grow more heads, so it loses some
in the process". To test your hypothesis you decide to cut off one final head. This time Niebla has simply loses the head
you cut off. \\

\noindent It would seem your enemy Niebla is an unstable beast, it seems to grow or decrease its head count almost at
random. But one thing you do notice is that there is a one-to-one correspondance to the number of heads Niebla has before
and after you decide to attack (i.e. Niebla will never enter a state where it has $x$ heads and suddenly go to either
$y$ heads or $z$ heads where $y \neq z$). This is great! It just might be possible to finish your long journey here
and now! \\

\noindent Unforunately you're on a timer, you only have a limited amount of health points left and for each turn where
you decide to defend yourself from Niebla's attack you lose 1 health point, and when you decide to take the brunt of the
Niebla's attack you take damage equivalent to the current number of heads that Niebla has. As previously dicussed Niebla
always follows a set of rules when determing it's head count after a set of moves. In the cases where a rule is not
specified for a certain head cout, you may assume Niebla will only lose 1 head if you attack. Furthermore, Niebla will
keep on attack you on each turn until it has reached 0 heads, or you have reached 0 or less health. \\

\noindent So you must choose quickly if it is possible to defeat Niebla or to retreat to your army and face Niebla
some other day.\\

\section*{Input}

The first line of input conists of three integers, $1 \leq N, H, R \leq 5000$, representing the current number of heads
Nieblas has, the current amount of health you have, and the amount of rules that Niebla follows with respect to its
head count. Following this will be $R$ lines of input consisting two integers $1 \leq n_i , n_j \leq 5000$ representing one
rule Niebla must always follows when determining it's head count. You may assume that Niebla's head count will never
exceed 5000. \\

\section*{Output}
Should it be impossible to defeat Niebla, you should print the phrase "retreat", otherwise we would like to know the
minimum such sequences of moves required to defeat Niebla. \\
